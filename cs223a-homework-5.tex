%insert copyright here
\documentclass{exam}
\usepackage{mathtools}
\begin{document}
\title{Stanford CS223A - Introduction to robotics \\
  Homework \#5}
\author{Arn-O}
\date{April 2014}
\maketitle

*** clean subscript and uperscript with ... ***

\begin{questions}

\question

\begin{parts}

\part

This question is not trial and the notes are sufficient to answer to it. You have to consider that the angular momentum and the angular velocity are vectors, and then describe the relationship between them.

The standard expression for angular momentum is:

\begin{subequations}
  \begin{equation}
    \Phi = I \omega
  \end{equation}
  \begin{equation}
    \Phi' = I \omega'
  \end{equation}
\end{subequations}

Let's consider now that they are vectors. That gives the following set of equations:

\begin{subequations}
  \begin{equation}
    \Phi' = R \Phi
  \end{equation}
  \begin{equation}
    \omega' = R \omega
  \end{equation}
\end{subequations}

Where ${R}$ is the rotation matrix between ${C}$ and ${C'}$. By combining the equations:

\begin{subequations}
  \begin{equation}
    \Phi' = R \Phi = R I \omega = I' \omega' = I' R \omega
  \end{equation}
  \begin{equation}
    I' = R I R^{-1}
  \end{equation}
\end{subequations}

We can now apply the parallel axis theorem from the page 4 of the handout 7, between the intermediate frame and the frame ${A}$.

\begin{equation}
  I_A = I_C' + m[(p_C^T p_C)I_3 - p_C p_C^T]
\end{equation} 

So finally:

\begin{equation}
  I_{A} = RI_{C}R^{-1} + m[(p_{C}^{T} p_{C})I_{3} - p_{C} p_{C}^{T}]
\end{equation} 

\part

Let's calculate ${I_{C}}$, considering that ${s_{x}=4}$, ${s_{y}=6}$ and ${s_{z}=2}$.

\begin{equation}
  I_{C} = \begin{bmatrix}
    40 & 0  & 0  \\
    0  & 20 & 0  \\
    0  & 0  & 52 \\
          \end{bmatrix}
\end{equation}

\part

${I_{A}}$ will be calculated term by term. ${p_{C} = [1, 1, 2]^{T}}$.

So for the first term:

\begin{equation}
  RI_{C}R^{-1} = \begin{bmatrix}
    30 & 10 & 0  \\
    10 & 30 & 0  \\
    0  & 0  & 52 \\
                 \end{bmatrix}
\end{equation}

For the second term, ${p_{C}^{T}p_{C} = 6}$.

That gives:

\begin{equation}
  m(p_{C}^{T}p_{C})I_{3} = 
    \begin{bmatrix}
      72 & 0  & 0  \\
      0  & 72 & 0  \\
      0  & 0  & 72 \\
    \end{bmatrix}
\end{equation}

And for the last term:

\begin{equation}
  m(p_{C}^{T}p_{C}) = 
    \begin{bmatrix}
      12 & 12 & 24 \\
      12 & 12 & 24 \\
      24 & 24 & 48 \\
    \end{bmatrix}
\end{equation}

Final result:

\begin{equation}
  I_{A} =
    \begin{bmatrix}
      90  & -2  & -24 \\
      -2  & 90  & -24 \\
      -24 & -24 & 76  \\
    \end{bmatrix}
\end{equation}

\end{parts}

\question

\begin{parts}

\part

\begin{equation}
  \prescript{0}{C_{1}}T =
    \begin{bmatrix}
      c\theta & s\theta  & 0 & \frac{L_{1}}{2}c\theta \\
      s\theta & -c\theta & 0 & \frac{L_{1}}{2}s\theta \\
      0       & 0        & 1 & 0                      \\
      0       & 0        & 0 & 1                      \\
    \end{bmatrix}
\end{equation}

\begin{equation}
  \prescript{0}{C_{2}}T =
    \begin{bmatrix}
      c\theta & s\theta  & 0 & L_{1}c\theta \\
      s\theta & -c\theta & 0 & L_{1}s\theta \\
      0       & 0        & 1 & d_{2}        \\
      0       & 0        & 0 & 1            \\
    \end{bmatrix}
\end{equation}

\part

\begin{equation}
  \prescript{0}{}J_{v_{1}} =
    \begin{bmatrix}
      -\frac{L_{1}}{2}s\theta & 0 \\
      \frac{L_{1}}{2}c\theta  & 0 \\
      0                       & 0 \\
    \end{bmatrix}
\end{equation}

\begin{equation}
  \prescript{0}{}J_{v_{2}} =
    \begin{bmatrix}
      -L_{1}s\theta & 0 \\
      L_{1}c\theta   & 0 \\
      0             & 1 \\
    \end{bmatrix}
\end{equation}

\part

\begin{equation}
  \prescript{0}{}J_{\omega_{1}} =
    \begin{bmatrix}
      0 & 0 \\
      0 & 0 \\
      1 & 0 \\
    \end{bmatrix}
\end{equation}

\begin{equation}
  \prescript{0}{}J_{\omega_{2}} =
    \begin{bmatrix}
      0 & 0 \\
      0 & 0 \\
      1 & 0 \\
    \end{bmatrix}
\end{equation}

There is no angular velocity since the second joint is prismatic.

\part

For the link 1:

\begin{subequations}
  \begin{equation}
    s_{x} = L_1
  \end{equation}
  \begin{equation}
    s_{y} = h
  \end{equation}
  \begin{equation}
    s_{z} = h
  \end{equation}
\end{subequations}

\begin{equation}
  \prescript{C_1}{}I_1 = \begin{bmatrix}
    \frac{m_1}{12}2h^2                         & 0             & 0           \\
    0                & \frac{m_1}{12}(L_1^2 + h^2)             & 0           \\
    0                & 0                       & \frac{m_1}{12}(L_1^2 + h^2) \\
                         \end{bmatrix}
\end{equation}

For the link 2:

\begin{subequations}
  \begin{equation}
    s_{x} = h
  \end{equation}
  \begin{equation}
    s_{y} = h
  \end{equation}
  \begin{equation}
    s_{z} = L_2
  \end{equation}
\end{subequations}

\begin{equation}
  \prescript{C_2}{}I_2 = \begin{bmatrix}
    \frac{m_2}{12}(L_2^2 + h^2)                & 0             & 0           \\
    0                & \frac{m_2}{12}(L_2^2 + h^2)             & 0           \\
    0                & 0                       & \frac{m_2}{12}2h^2          \\
                         \end{bmatrix}
\end{equation}

\part

Let's break the calculation down term by term.

\begin{equation}
  J_{v1}^T J_{v1} = \begin{bmatrix}
    \frac{L_1}{4} & 0 \\
    0             & 0 \\
                    \end{bmatrix}
\end{equation}

\begin{equation}
  J_{v2}^T J_{v2} = \begin{bmatrix}
    \L_1^2 & 0 \\
    0      & 1 \\
                    \end{bmatrix}
\end{equation}

\begin{equation}
  J_{\omega1}^T \prescript{C_1}{}I_1 J_{\omega1} =
    \begin{bmatrix}
      I_{zz1} & 0 \\
      0       & 0 \\
    \end{bmatrix}
\end{equation}

\begin{equation}
  J_{\omega2}^T \prescript{C_2}{}I_2 J_{\omega2} =
    \begin{bmatrix}
      I_{zz2} & 0 \\
      0       & 0 \\
    \end{bmatrix}
\end{equation}

So, finally, the mass matrix is:

\begin{equation}
  M = \begin{bmatrix}
    m_1\frac{L_1^2}{4} + m_{2}L_1^2 + I_{zz1} + I_{zz2} & 0   \\
    0                                                   & m_2 \\
      \end{bmatrix}
\end{equation}

The first term can now be developed:

\begin{subequations}
  \begin{equation}
    M_{11} = L_1^2\frac{m_1}{4} + L_1^2m_2 + L_1^2\frac{m_1}{12} +
             h^2\frac{m_1}{12} + h^2\frac{m_2}{6}
  \end{equation}
  \begin{equation}
    M_{11} = L_1^2\frac{m_1}{3} + h^2\frac{m_1}{12} + L_1^2m_2 + 
             h^2\frac{m_2}{6}
  \end{equation}
\end{subequations}

\part

The joints positions do not affect the centrifugal and inertia along the first joint axis, so at the end, the result will be null. Let's try to develop the equation anyhow.

\begin{equation}
  v = \begin{bmatrix}
        \dot m_{11} & 0          \\
	0          & \dot m_{22} \\
      \end{bmatrix} \dot q 
      - \frac{1}{2} 
      \begin{bmatrix}
        \dot q^T \begin{pmatrix}
		   m_{111} & 0       \\
		   0       & m_{221} \\
		 \end{pmatrix} \dot q    \\
                                         \\
	  \dot q^T \begin{pmatrix}
		   m_{112} & 0       \\
		   0       & m_{222} \\
		 \end{pmatrix} \dot q
      \end{bmatrix}
\end{equation}

Let's break this equation down.

\begin{equation}
  \dot m_{11} = \frac{dm_{11}}{dt} 
              = \frac{dm_{11}}{dq_{1}} \times \frac{dq_1}{dt} +
	        \frac{dm_{11}}{dq_{2}} \times \frac{dq_2}{dt}
	      = m_{111}\dot q_1 + m_{112}\dot q_2
\end{equation}

In the same way, ${\dot m_{22} = m_{221}\dot q_1 + m_{222}\dot q_2}$.

\begin{equation}
  \begin{bmatrix}
    m_{111}\dot q_1 + m_{112}\dot q_2 & 0          \\
    0 &        m_{221}\dot q_1 + m_{222}\dot q_2   \\
  \end{bmatrix} 
  \begin{bmatrix}
    \dot q_1 \\
    \dot q_2 \\
  \end{bmatrix} =
  \begin{bmatrix}
    m_{111} \dot q_1^2 + m_{112} \dot q_1 \dot q_2 \\
    m_{221} \dot q_1 \dot q_2 + m_{222} \dot q_2^2 \\
  \end{bmatrix}
\end{equation}

\begin{equation}
  \dot q^T \begin{bmatrix}
     m_{111} & 0       \\
     0       & m_{221} \\
           \end{bmatrix} \dot q \\ 
  = [m_{111} \dot q_1^2 + m_{221} \dot q_2^2]
\end{equation}

Which gives for the second term:

\begin{equation}
  \dot q^T \begin{bmatrix}
     m_{112} & 0       \\
     0       & m_{222} \\
           \end{bmatrix} \dot q \\ 
  = [m_{112} \dot q_1^2 + m_{222} \dot q_2^2]
\end{equation}

\begin{equation}
  v = \begin{bmatrix}
    m_{111} \dot q_1^2 -\frac{1}{2}m_{111} \dot q_1^2 
    -\frac{1}{2}m_{221} \dot q_2^2 + m_{112} \dot q_1 \dot q_2 \\
    -\frac{1}{2}m_{112} \dot q_1^2 -\frac{1}{2}m_{222} \dot q_2^2
    +m_{222} \dot q_2^2 + m_{221} \dot q_1 \dot q_2
      \end{bmatrix}
\end{equation}

From this last equation, we can define ${C}$ and ${B}$:

\begin{subequations}
  \begin{equation}
    C(q) = \begin{bmatrix}
	     \frac{1}{2}m_{111}  & -\frac{1}{2}m_{221} \\
	     -\frac{1}{2}m_{112} & \frac{1}{2}m_{222}  \\
	   \end{bmatrix}
  \end{equation}
  \begin{equation}
    B(q) = \begin{bmatrix}
	     m_{112} \\
	     m_{221} \\
	   \end{bmatrix}
  \end{equation}
\end{subequations}

\part

${C(q) = [0]}$ and ${B(q) = [0]}$.

\end{parts}

\end{questions}

\end{document}

