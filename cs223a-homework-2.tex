%insert copyright here
\documentclass{exam}
\usepackage{mathtools}
\begin{document}
\title{Stanford CS223A - Introduction to robotics \\
  Homework \#2}
\author{Arn-O}
\date{January 2014}
\maketitle

\begin{questions}

\question

\begin{parts}
\part

Quite a complex question that learns me the similarity transforms. The first really tricky point is that after a $T_{1}$ displacement, the frame transform is $T_{1}^{-1}$. So the transform between the frame $\{0\}$ and the frame $\{1\}$ can be written:

\begin{equation}
  P\{1\} = T_{1}^{-1}P\{0\}
\end{equation}

By definition, after the first displacement:

\begin{equation}
  P_{1}\{0\} = T_{1}P\{0\}
\end{equation}

And after the second displacement:

\begin{equation}
  P_{2}\{1\} = T_{2}P_{1}\{1\}
\end{equation}
 
And the result is:

\begin{equation}
  P_{2}\{0\} = T'_{2}P\{0\}
\end{equation}

The second displacement explained in the frame $\{0\}$ gives:

\begin{equation}
  T_{1}^{-1}P_{2}\{0\} = T_{2}T_{1}^{-1}P\{0\}
\end{equation}

And finally:

\begin{equation}
  T'_{2} = T_{1}T_{2}T_{1}^{-1}
\end{equation}

This could also be derivate from the similarity transform equation:

\begin{equation}
  B = T^{-1}AT
\end{equation}

Where $T$ is the frame transform from $\{A\}$ to $\{B\}$. In this case:

\begin{equation}
  T = T_{1}^{-1}
\end{equation}

\end{parts}

\end{questions}

\end{document}

