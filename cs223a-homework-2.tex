%insert copyright here
\documentclass{exam}
\usepackage{mathtools}
\begin{document}
\title{Stanford CS223A - Introduction to robotics \\
  Homework \#2}
\author{Arn-O}
\date{January 2014}
\maketitle

\begin{questions}

\question

\begin{parts}
\part

Quite a complex question that learns me the similarity transforms. The first really tricky point is that after a $T_{1}$ displacement, the frame transform is $T_{1}^{-1}$. So the transform between the frame $\{0\}$ and the frame $\{1\}$ can be written:

\begin{equation}
  P\{1\} = T_{1}^{-1}P\{0\}
\end{equation}

By definition, after the first displacement:

\begin{equation}
  P_{1}\{0\} = T_{1}P\{0\}
\end{equation}

And after the second displacement:

\begin{equation}
  P_{2}\{1\} = T_{2}P_{1}\{1\}
\end{equation}
 
And the result is:

\begin{equation}
  P_{2}\{0\} = T'_{2}P\{0\}
\end{equation}

The second displacement explained in the frame $\{0\}$ gives:

\begin{equation}
  T_{1}^{-1}P_{2}\{0\} = T_{2}T_{1}^{-1}P\{0\}
\end{equation}

And finally:

\begin{equation}
  T'_{2} = T_{1}T_{2}T_{1}^{-1}
\end{equation}

This could also be derivate from the similarity transform equation:

\begin{equation}
  B = T^{-1}AT
\end{equation}

Where $T$ is the frame transform from $\{A\}$ to $\{B\}$. In this case:

\begin{equation}
  T = T_{1}^{-1}
\end{equation}

\part

In this case, we use the similarity transform with $T_{3}$ after a $T'_{2}T_{1}$ displacement:

\begin{equation}
  T'_{3} = (T'_{2}T_{1})T_{3}(T'_{2}T_{1})^{-1}
\end{equation}

So with the result of the first part:

\begin{equation}
  T'_{3} = T_{1}T_{2}T_{3}T_{1}^{-1}T_{2}^{-1}
\end{equation}

\part

The same formulae is applied again on this joint:

\begin{equation}
  T'_{4} = (T'_{3}T'_{2}T_{1})T_{4}(T'_{3}T'_{2}T_{1})^{-1}
\end{equation}

So if we develop this equation:

\begin{equation}
  T'_{4} = (T_{1}T_{2}T_{3}T_{1}^{-1}T_{2}^{-1}
            T_{1}T_{2}T_{1}^{-1}
            T_{1})
	    T_{4}
           (T_{1}T_{2}T_{3}T_{1}^{-1}T_{2}^{-1}
	    T_{1}T_{2}T_{1}^{-1}
	    T_{1})^{-1}
\end{equation}

After the simplification:

\begin{equation}
  T'_{4} = (T_{1}T_{2}T_{3})T_{4}(T_{3}^{-1}T_{2}^{-1}T_{1}^{-1})
\end{equation}

\part

From the fixed to the free end, the displacement is:

\begin{equation}
  T = T'_{4}T'_{3}T'_{2}T'_{1}
\end{equation}

Which gives:

\begin{equation}
  T = T_{1}T_{2}T_{3}T_{4}T_{3}^{-1}T_{2}^{-1}T_{1}^{-1}
      T_{1}T_{2}T_{3}T_{1}^{-1}T_{2}^{-1}
      T'_{2}T'_{1}
\end{equation}

\begin{equation}
  T = T_{1}T_{2}T_{3}T_{4}T_{1}^{-1}T_{2}^{-1}
      T_{1}T_{2}T_{1}^{-1}    
      T_{1}
\end{equation}

\begin{equation}
  T = T_{1}T_{2}T_{3}T_{4}
\end{equation}

And we come to the same result.

\end{parts}

\question

\begin{parts}
\part

I did not find any easy way to draw a schematic in LaTeX.

Just a couple of comments about the choice of the frames:

\begin{itemize}
  \item the frames $\{0\}$, $\{1\}$ and $\{2\}$ coincide
  \item the axis of the joint 1 and 2 intersect, so the $x_{1}$ axis is placed freely
  \item an extra parameter is required between the joint 2 and 3
\end{itemize}

\part

The Denavit-Hartenberg parameters are the following:

\begin{centering}

\begin{tabular}{|| c | c | c | c | c ||}
  \hline
  i & $a_{i-1}$ & $\alpha_{i-1}$ & $d_{i}$ & $\theta_{i}$ \\
  \hline
  1 & 0         & 0              & 0       & $\theta_{1}$ \\
  \hline
  2 & 0         & +90            & 0       & $\theta_{2}$ \\
  \hline
  3 & $l$       & +90            & $d_{3}$ & 0            \\
  \hline
\end{tabular}

\end{centering}

\part

Let's apply the forward kinematics formula from the DH parameters:

\begin{equation}
  \prescript{i-1}{i}T = \begin{bmatrix}
    c\theta_{i}              & -s\theta_{i}             & 0              & a_{i-1}             \\
    s\theta_{i}c\alpha_{i-1} & c\theta_{i}c\alpha_{i-1} & -s\alpha_{i-1} & -s\alpha_{i-1}d_{i} \\
   s\theta_{i}s\alpha_{i-1}  & c\theta_{i}s\alpha_{i-1} & c\alpha_{i-1}  & c\alpha_{i-1}d_{i}  \\
   0                         & 0                        & 0              & 1                   \\ 
			\end{bmatrix}
\end{equation}
		  
\end{parts}

\end{questions}

\end{document}

