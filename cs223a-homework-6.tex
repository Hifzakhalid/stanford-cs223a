%insert copyright here
\documentclass{exam}
\usepackage{mathtools}
\begin{document}
\title{Stanford CS223A - Introduction to robotics \\
  Homework \#6}
\author{Arn-O}
\date{May 2014}
\maketitle

\begin{questions}

\question

\begin{parts}

\part

From the handout, when the equation of motion is ${m \ddot x + b \dot x + kx = 0}$, the natural frequency and damping are given by:

\begin{subequations}
  \begin{equation}
    \omega_n = \sqrt{ \frac{k}{m} }
  \end{equation}
  \begin{equation}
    \xi_n = \frac{b}{2\sqrt{km}}
  \end{equation}
\end{subequations}

So in the case of the exercise:

\begin{subequations}
  \begin{equation}
    \omega_n = \sqrt{ \frac{25}{4} }
  \end{equation}
  \begin{equation}
    \xi_n = 1
  \end{equation}
\end{subequations}

This system is critically damped.

\part

The equation of motion of the system is now:

\begin{equation}
  4 \ddot x + (20 + k_v) \dot x + (25 + k_p) x = 0
\end{equation}

The stiffness is 36, so:

\begin{subequations}
  \begin{equation}
    25 + k_p = 36
  \end{equation}
  \begin{equation}
    k_p = 11
  \end{equation}
\end{subequations}

The system stay critically damped, so:

\begin{subequations}
  \begin{equation}
    \frac{(20 + k_v)}{2 \sqrt{36 \times 4}} = 1
  \end{equation}
  \begin{equation}
    k_v = 4
  \end{equation}
\end{subequations}

At rest, the position is 0.

\part 

${\alpha}$ estimates the mass, in this case it is just the mass:

\begin{equation}
  \alpha = \hat{m} = 4
\end{equation}

\begin{equation}
  \beta = 30 sign(\dot x) + 25 x
\end{equation}

Using the error notation, the equation of motion becomes:

\begin{equation}
  \alpha \ddot e + \alpha k'_v \dot e + \alpha k'_p e = 0
\end{equation}

Keeping the stiffness at 36 implies:

\begin{subequations}
  \begin{equation}
    \alpha k'_p = 36
  \end{equation}
  \begin{equation}
    k'_p = 9
  \end{equation}
\end{subequations}

For a critically damped system:

\begin{subequations}
  \begin{equation}
    \frac{(\alpha + k'_v)}{2 \sqrt{\alpha k'_p m}} = 1
  \end{equation}
  \begin{equation}
    k'_v = 2 \sqrt{k'_p} = 6
  \end{equation}
\end{subequations}

\part

For a steady state:

\begin{subequations}
  \begin{equation}
    \alpha k'_p e_{dist} = f_{dist}
  \end{equation}
  \begin{equation}
    e_{dist} = \frac{f_{dist}}{\alpha k'_p} = 0.111
  \end{equation}
\end{subequations}

\end{parts}

\question

\begin{parts}

\part

If the joint 2 is locked, ${\ddot \theta_2 = \dot \theta_2 = 0}$. The equations of motion and the equation of the control give:

\begin{equation}
  (4 + c_2) \ddot \theta_1 + 40 \dot \theta_1 
    + 400 (\theta_1 - \theta_{1d}) = 0
\end{equation}

Given that ${-1 \leq c_2 \leq 1}$, the inertia perceived by the joint 1 is between 3 and 5.

That gives the following frequencies:

\begin{subequations}
  \begin{equation}
    \omega_{min} = \sqrt{ \frac{400}{5} } = 8.944
  \end{equation}
  \begin{equation}
    \omega_{max} = \sqrt{ \frac{400}{3} } = 11.547
  \end{equation}
\end{subequations}

\part

The minimum ratio is for ${\theta_2 = 0^{\circ}}$ and the maximum for ${\theta_2 = 180^{\circ}}$.

\begin{subequations}
  \begin{equation}
    \xi_{min} = \frac{40}{2 \sqrt{400 \times 5}} = 0.447
  \end{equation}
  \begin{equation}
    \xi_{max} = \frac{40}{2 \sqrt{400 \times 3}} = 0.577
  \end{equation}
\end{subequations}

\end{parts}

\end{questions}

\end{document}

