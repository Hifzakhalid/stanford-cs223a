%insert copyright here
\documentclass{exam}
\usepackage{mathtools}
\begin{document}
\title{Stanford CS223A - Introduction to robotics \\
  Homework \#6}
\author{Arn-O}
\date{May 2014}
\maketitle

\begin{questions}

\question

\begin{parts}

\part

From the handout, when the equation of motion is ${m \ddot x + b \dot x + kx = 0}$, the natural frequency and damping are given by:

\begin{subequations}
  \begin{equation}
    \omega_n = \sqrt{ \frac{k}{m} }
  \end{equation}
  \begin{equation}
    \xi_n = \frac{b}{2\sqrt{km}}
  \end{equation}
\end{subequations}

So in the case of the exercise:

\begin{subequations}
  \begin{equation}
    \omega_n = \sqrt{ \frac{25}{4} }
  \end{equation}
  \begin{equation}
    \xi_n = 1
  \end{equation}
\end{subequations}

This system is critically damped.

\part

The equation of motion of the system is now:

\begin{equation}
  4 \ddot x + (20 + k_v) \dot x + (25 + k_p) x = 0
\end{equation}

The stiffness is 36, so:

\begin{subequations}
  \begin{equation}
    25 + k_p = 36
  \end{equation}
  \begin{equation}
    k_p = 11
  \end{equation}
\end{subequations}

The system stay critically damped, so:

\begin{subequations}
  \begin{equation}
    \frac{(20 + k_v)}{2 \sqrt{36 \times 4}} = 1
  \end{equation}
  \begin{equation}
    k_v = 4
  \end{equation}
\end{subequations}

At rest, the position is ${0}$.

\part 

\end{parts}

\end{questions}

\end{document}

