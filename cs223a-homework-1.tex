%insert copyright here
\documentclass{article}
\usepackage{mathtools}
\begin{document}
\title{Stanford CS223A - Introduction to robotics \\
  Homework \#1}
\author{Arn-O}
\date{December 2013}
\maketitle

Exercise 1.

Short reminder of the naming conventions:

\begin{equation}
  \prescript{A}{}{P} = \prescript{A}{B}{R} \prescript{B}{}{P}
\end{equation}

The problem described is a combination of 2 rotations in two different frames:

\begin{equation}
  \prescript{A}{B}{R} = \prescript{A}{A'}{R} \prescript{A'}{B}{R} = R_{y}(\theta) R_{z}(\phi)
\end{equation}

The first rotation can be described as a basic rotation matrix:

\begin{equation}
  \prescript{A}{A'}{R} = R_{y}(\theta) = \begin{bmatrix}
                                            c\theta  & 0 & s\theta \\
                                            0        & 1 & 0        \\
                                            -s\theta & 0 & c\theta
                                          \end{bmatrix}
\end{equation}

The second rotation is also a basic rotation matrix:

\begin{equation}
  \prescript{A'}{B}{R} = R_{z}(\phi) = \begin{bmatrix}
                                          c\phi & -s\phi & 0 \\
                                          s\phi & c\phi  & 0 \\
                                          0     & 0      & 1
                                        \end{bmatrix}
\end{equation}

\end{document}

