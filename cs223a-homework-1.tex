%insert copyright here
\documentclass{exam}
\usepackage{mathtools}
\begin{document}
\title{Stanford CS223A - Introduction to robotics \\
  Homework \#1}
\author{Arn-O}
\date{December 2013}
\maketitle

\begin{questions}

\question

Equation giving the vector representation between frame A and B:

\begin{equation}
  \prescript{A}{}{P} = \prescript{A}{B}{R} \prescript{B}{}{P}
\end{equation}

The problem described is a combination of rotations of two different frames:

\begin{equation}
  \prescript{A}{B}{R} = \prescript{A}{A'}{R} \prescript{A'}{B}{R} = R_{y}(\theta) R_{z}(\phi)
\end{equation}

The first rotation can be described as a basic rotation matrix around the y axis:

\begin{equation}
  \prescript{A}{A'}{R} = R_{y}(\theta) = \begin{bmatrix}
                                            c\theta  & 0 & s\theta \\
                                            0        & 1 & 0        \\
                                            -s\theta & 0 & c\theta
                                          \end{bmatrix}
\end{equation}

The second rotation is a basic rotation matrix around the z axis:

\begin{equation}
  \prescript{A'}{B}{R} = R_{z}(\phi) = \begin{bmatrix}
                                          c\phi & -s\phi & 0 \\
                                          s\phi & c\phi  & 0 \\
                                          0     & 0      & 1
                                        \end{bmatrix}
\end{equation}

So the composition of both rotations gives:

\begin{equation}
  \prescript{A}{B}{R} = \begin{bmatrix}
                          c\theta c\phi  & -c\theta s\phi & s\theta \\
                          s\phi          & c\phi          & 0       \\
                          -s\theta c\phi & -s\theta s\phi & c\theta
                        \end{bmatrix}
\end{equation}

This exercise is the case of a frame representation using Euler notation. The reference frame is required in the notations.

\question

The first rotation gives:

\begin{equation}
  \prescript{A}{}{P'} = R_{z}(\phi) \prescript{A}{}{P}
\end{equation}

The second rotation is then applied to the first one:

\begin{equation}
  \prescript{A}{}{P''} = R_{y}(\theta) \prescript{A}{}{P'} = R_{y}(\theta) R_{z}(\phi) \prescript{A}{}{P}
\end{equation}

So:

\begin{equation}
  R(\phi, \theta) = R_{y}(\theta) R_{z}(\phi)
\end{equation}

The matrix is the same. This vector can imagined as a reference frame, so both exercise demonstrates that a Z-Y fixed angle notation is equivalent to a Y-Z Euler angle notation.

\question

\begin{parts}
\part

Here is the equation to compute the transformation from frame $\{A\}$ to frame $\{B\}$:

\begin{equation}
  \prescript{A}{B}{T} = \prescript{B}{A}{T}^{T} \\
  = \begin{bmatrix}
      R^{T} & -R^{T}Q \\
      0     & 1       
    \end{bmatrix}
\end{equation}

\begin{equation}
  R^{T} = \begin{bmatrix}
            1 & 0        & 0       \\
            0 & c\theta  & s\theta \\
	    0 & -s\theta & c\theta
	  \end{bmatrix}
\end{equation}

\begin{equation}
  R^{T}Q = \begin{bmatrix}
             1                    \\
             2c\theta + 3s\theta  \\
	     -2s\theta + 3c\theta \\
	   \end{bmatrix}
\end{equation}

By combining the previous matrix and vector into an homogenous transform:

\begin{equation}
  \prescript{A}{B}{T} = \begin{bmatrix}
                          1 & 0        & 0       & -1                   \\
			  0 & c\theta  & s\theta & -2c\theta - 3s\theta \\
			  0 & -s\theta & c\theta & 2s\theta - 3 c\theta \\
			  0 & 0        & 0       & 1                    \\
			\end{bmatrix}
\end{equation}

\part

Considering:

\begin{equation}
  cos(45) = sin(45) = \frac{1}{\sqrt{2}}
\end{equation} 

The transformation matix becomes:

\begin{equation}
  \prescript{A}{B}{T} =
    \begin{bmatrix}
       1 & 0                   & 0                  & -1                  \\
       0 & \frac{1}{\sqrt{2}}  & \frac{1}{\sqrt{2}} & \frac{-5}{\sqrt{2}} \\
       0 & \frac{-1}{\sqrt{2}} & \frac{1}{\sqrt{2}} & \frac{-1}{\sqrt{2}} \\
       0 & 0                   & 0                  & 1                   \\
    \end{bmatrix}
\end{equation}

Applied to the given vector:

\begin{equation}
  \prescript{A}{}{P} = \begin{bmatrix}
                         3                  \\
                         \frac{6}{\sqrt{2}} \\
                         0                  \\
                       \end{bmatrix}
		     = \begin{bmatrix}
		         3    \\
			 4.24 \\
			 0    \\
	               \end{bmatrix}
\end{equation}

\end{parts}

\question

\begin{parts}
\part

The matrix is orthogonal:

\begin{equation}
  {R}^{T}R = I
\end{equation}

And the determinant is equal to 0: 

\begin{equation}
  \frac{1}{4} + 0 + \frac{1}{4} - \frac{-1}{4} - \frac{-1}{4} - 0 = 1 
\end{equation}

So this matrix is a rotation matrix.

\part

The angle is given by the formula:

\begin{equation}
  \theta = \cos^{-1}(\frac{r_{11} + r_{22} + r_{33} - 1}{2})
\end{equation}

\begin{equation}
  \theta = \cos^{-1}(\frac{ 4 - \sqrt{2} }{ 4 \sqrt{2} })
\end{equation}

\begin{equation}
  \theta = 62.8 \deg
\end{equation}

\end{parts}

\end{questions}

\end{document}

