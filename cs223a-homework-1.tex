%insert copyright here
\documentclass{article}
\usepackage{mathtools}
\begin{document}
\title{Stanford CS223A - Introduction to robotics \\
  Homework \#1}
\author{Arn-O}
\date{December 2013}
\maketitle

Exercise 1.

Equation giving the vector representation between frame A and B:

\begin{equation}
  \prescript{A}{}{P} = \prescript{A}{B}{R} \prescript{B}{}{P}
\end{equation}

The problem described is a combination of rotations of two different frames:

\begin{equation}
  \prescript{A}{B}{R} = \prescript{A}{A'}{R} \prescript{A'}{B}{R} = R_{y}(\theta) R_{z}(\phi)
\end{equation}

The first rotation can be described as a basic rotation matrix around the y axis:

\begin{equation}
  \prescript{A}{A'}{R} = R_{y}(\theta) = \begin{bmatrix}
                                            c\theta  & 0 & s\theta \\
                                            0        & 1 & 0        \\
                                            -s\theta & 0 & c\theta
                                          \end{bmatrix}
\end{equation}

The second rotation is a basic rotation matrix around the z axis:

\begin{equation}
  \prescript{A'}{B}{R} = R_{z}(\phi) = \begin{bmatrix}
                                          c\phi & -s\phi & 0 \\
                                          s\phi & c\phi  & 0 \\
                                          0     & 0      & 1
                                        \end{bmatrix}
\end{equation}

So the composition of both rotations gives:

\begin{equation}
  \prescript{A}{B}{R} = \begin{bmatrix}
                          c\theta c\phi  & -c\theta s\phi & s\theta \\
                          s\phi          & c\phi          & 0 \\
                          -s\theta c\phi & -s\theta s\phi & c\theta
                        \end{bmatrix}
\end{equation}

This exercise is the case of a frame representation using Euler notation. The reference frame is required in the notations.

Exercise 2.

The first rotation gives:

\begin{equation}
  \prescript{A}{}{P'} = R_{z}(\phi) \prescript{A}{}{P}
\end{equation}

The second rotation is then applied to the first one:

\begin{equation}
  \prescript{A}{}{P''} = R_{y}(\theta) \prescript{A}{}{P'} = R_{y}(\theta) R_{z}(\phi) \prescript{A}{}{P}
\end{equation}

So:

\begin{equation}
  R(\phi, \theta) = R_{y}(\theta) R_{z}(\phi)
\end{equation}

The matrix is the same. This vector can imagined as a reference frame, so both exercise demonstrates that a Z-Y fixed angle notation is equivalent to a Y-Z Euler angle notation.

Exercise 3.


\end{document}

