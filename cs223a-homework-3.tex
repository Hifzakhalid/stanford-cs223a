%insert copyright here
\documentclass{exam}
\usepackage{mathtools}
\begin{document}
\title{Stanford CS223A - Introduction to robotics \\
  Homework \#3}
\author{Arn-O}
\date{January 2014}
\maketitle

\begin{questions}

\question

Let's deal with the linear velocity first:

\begin{equation}
  x_{P} = \begin{bmatrix}
            L_{1}c_{1} + L_{2}c_{1}c_{3} - s_{1}d_{2} \\
            L_{1}s_{1} + L_{2}s_{1}c_{3} + c_{1}d_{2} \\
            -L_{2}s_{3}                               \\
           \end{bmatrix}
\end{equation}

The derivation of this vector for each joint parameter gives:

\begin{equation}
  J_{V} = \begin{bmatrix}
    -L_{1}s_{1} - L_{2}s_{1}c_{3} - c_{1}d_{2} & -s_{1} & -L_{2}c_{1}s_{3} \\
    L_{1}c_{1} + L_{2}c_{1}c_{3} - s_{1}d_{2}  & c_{1}  & -L_{2}s_{1}s_{3} \\
    0                                          & 0      & -L_{2}c_{3}      \\
          \end{bmatrix}
\end{equation}

The angular velocity Jacobian can be determined using the trick of the intermediate end effector transform calculation. The missing transform being prismatic, it does not contribute 

\begin{equation}
  J_{w} = \begin{bmatrix}
    0 & 0 & -s_{1} \\
    0 & 0 & c_{1}  \\
    1 & 0 & 0      \\
          \end{bmatrix}
\end{equation}

\question
\begin{parts}

\part

The position of the origin can be determined using simple trigonometric equations:

\begin{equation}
  \prescript{0}{}P_{org} = \begin{bmatrix}
                             L_{2}c_{2}         \\
			     d_{1} + L_{2}s_{2} \\
			   \end{bmatrix}
\end{equation}

\part

Based on the previous equation, the linear velocity Jacobian is:

\begin{equation}
  J_{V} = \begin{bmatrix}
            0 & -L_{2}s_{1} \\
	    1 & L_{2}c_{2}  \\
	  \end{bmatrix}
\end{equation} 

\part

In case of singularity, $det(J_{V}) = 0$, meaning that $L_{2}s_{1} = 0$. In this case, $\theta_{2} = k\pi$. The end effector cannot move in the direction given by the $x$ axis.

\end{parts}

\end{questions}

\end{document}

