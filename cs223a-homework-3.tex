%insert copyright here
\documentclass{exam}
\usepackage{mathtools}
\begin{document}
\title{Stanford CS223A - Introduction to robotics \\
  Homework \#3}
\author{Arn-O}
\date{January 2014}
\maketitle

\begin{questions}

\question

Let's deal with the linear velocity first:

\begin{equation}
  x_{P} = \begin{bmatrix}
            L_{1}c_{1} + L_{2}c_{1}c_{3} - s_{1}d_{2} \\
            L_{1}s_{1} + L_{2}s_{1}c_{3} + c_{1}d_{2} \\
            -L_{2}s_{3}                               \\
           \end{bmatrix}
\end{equation}

The derivation of this vector for each joint parameter gives:

\begin{equation}
  J_{V} = \begin{bmatrix}
    -L_{1}s_{1} - L_{2}s_{1}c_{3} - c_{1}d_{2} & -s_{1} & -L_{2}c_{1}s_{3} \\
    L_{1}c_{1} + L_{2}c_{1}c_{3} - s_{1}d_{2}  & c_{1}  & -L_{2}s_{1}s_{3} \\
    0                                          & 0      & -L_{2}c_{3}      \\
          \end{bmatrix}
\end{equation}

The angular velocity Jacobian can be determined using the trick of the intermediate end effector transform calculation. The missing transform being prismatic, it does not contribute 

\begin{equation}
  J_{w} = \begin{bmatrix}
    0 & 0 & -s_{1} \\
    0 & 0 & c_{1}  \\
    1 & 0 & 0      \\
          \end{bmatrix}
\end{equation}

\question
\begin{parts}

\part

The position of the origin can be determined using simple trigonometric equations:

\begin{equation}
  \prescript{0}{}P_{org} = \begin{bmatrix}
                             L_{2}c_{2}         \\
			     d_{1} + L_{2}s_{2} \\
			   \end{bmatrix}
\end{equation}

\part

Based on the previous equation, the linear velocity Jacobian is:

\begin{equation}
  J_{V} = \begin{bmatrix}
            0 & -L_{2}s_{1} \\
	    1 & L_{2}c_{2}  \\
	  \end{bmatrix}
\end{equation} 

\part

In case of singularity, $det(J_{V}) = 0$, meaning that $L_{2}s_{1} = 0$. In this case, $\theta_{2} = k\pi$. The end effector cannot move in the direction given by the $x$ axis.

\end{parts}

\question
\begin{parts}

\part

All of the $x$ axis are aligned. The $z$ axis are directed to the top or out of the page. In this configuration, the D-H parameters are as follow:

\begin{centering}

\begin{tabular}{|| c | c | c | c | c ||}
  \hline
  i & $a_{i-1}$     & $\alpha_{i-1}$ & $d_{i}$      & $\theta_{i}$ \\
  \hline
  1 & 0             & 0              & $\theta_{1}$ & 0            \\
  \hline
  2 & $-90^{\circ}$ & 1              & $\theta_{2}$ & 0            \\
  \hline
  3 & $90^{\circ}$  & 1              & $\theta_{3}$ & 0            \\
  \hline
  4 & $-90^{\circ}$ & 1              & 0            & 0            \\
  \hline
\end{tabular}

\end{centering}

\part

We use the formula of the handout 3, p.4 to calculate the intermediate transforms.

\begin{equation}
  \prescript{0}{1}T = \begin{bmatrix}
    c\theta_{1} & -s\theta_{1} & 0 & 0 \\
    s\theta_{1} & c\theta_{1}  & 0 & 0 \\
    0           & 0            & 1 & 0 \\
    0           & 0            & 0 & 1 \\
                     \end{bmatrix}
\end{equation}

\begin{equation}
  \prescript{1}{2}T = \begin{bmatrix}
    c\theta_{2} & -s\theta_{2} & 0 & 1 \\
    0           & 0            & 1 & 0 \\
    -s\theta{2} & -c\theta_{2} & 0 & 0 \\
    0           & 0            & 0 & 1 \\
                     \end{bmatrix}
\end{equation}

\begin{equation}
  \prescript{2}{3}T = \begin{bmatrix}
    c\theta_{3} & -s\theta_{3} & 0  & 1 \\
    0           & 0            & -1 & 0 \\
    s\theta_{3}  & c\theta_{3}  & 0  & 0 \\
    0           & 0            & 0  & 1 \\
                     \end{bmatrix}
\end{equation}

\begin{equation}
  \prescript{3}{4}T = \begin{bmatrix}
    1 & 0  & 0 & 1 \\
    0 & 0  & 1 & 0 \\
    0 & -1 & 0 & 0 \\
    0 & 0  & 0 & 1 \\
                     \end{bmatrix}
\end{equation}

The combination of the intermediate transforms gives:

\begin{equation}
  \prescript{0}{2}T = \prescript{0}{1}T\prescript{1}{2} = \begin{bmatrix}
    c_{1}c_{2} & -c_{1}s_{2} & -s_{1} & c_{1} \\
    s_{1}c_{2} & -s_{1}s_{2} & c_{1}  & s_{1} \\
    -s_{2}     & -c_{2}      & 0      & 0     \\
    0          & 0           & 0      & 1     \\
                                                          \end{bmatrix}
\end{equation}

\begin{equation}
  \prescript{0}{3}T = \prescript{0}{2}T\prescript{2}{3} = \begin{bmatrix}
    c_{1}c_{2}c_{3}-s_{1}s_{3} & -c_{1}c_{2}s_{3}-s_{1}c_{3} & c_{1}s_{2} & c_{1}c_{2}+c_{1} \\
    s_{1}c_{2}c_{3}+c_{1}s_{3} & -s_{1}c_{2}s_{3}+c_{1}c_{3} & s_{1}s_{2} & s_{1}c_{2}+s_{1} \\
    -s_{2}c_{3}                & s_{2}s_{3}                  & c_{2}      & -s_{2}           \\
    0                          & 0                           & 0          & 1                \\
                                                          \end{bmatrix}
\end{equation}

\begin{equation}
  \prescript{0}{4}T = \prescript{0}{3}T\prescript{3}{4} = \begin{bmatrix}
    c_{1}c_{2}c_{3}-s_{1}s_{3} & -c_{1}s_{2} & -c_{1}c_{2}s_{3}-s_{1}c_{3} & c_{1}c_{2}c_{3}-s_{1}s_{3}+c_{1}c_{2}+c_{1} \\
    s_{1}c_{2}c_{3}+c_{1}s_{3} & -s_{1}s_{2} & s_{1}c_{2}s_{3}+c_{1}c_{3}  & s_{1}c_{2}c_{3}+c_{1}s_{3}+s_{1}c_{2}+s_{1} \\
    -s_{2}c_{3}                & -c_{2}      &  s_{2}s_{3}                 & -s_{2}c_{3}-s_{2}                           \\
    0                          & 0           & 0                           & 1                                           \\
                                                          \end{bmatrix}
\end{equation}


\end{parts}

\end{questions}

\end{document}

