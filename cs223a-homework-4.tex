%insert copyright here
\documentclass{exam}
\usepackage{mathtools}
\begin{document}
\title{Stanford CS223A - Introduction to robotics \\
  Homework \#4}
\author{Arn-O}
\date{February 2014}
\maketitle

\begin{questions}

\question

\begin{parts}

\part

The linear Jacobian is derivated from the vector position. To be noticed the following trigonometric properties:

\begin{equation}
  \frac{\partial s_{12}}{\partial \theta_{1}} =
    c_{1}c_{2}-s_{1}s_{2} = c_{12}
\end{equation}

And:

\begin{equation}
  \frac{\partial c_{12}}{\partial \theta_{1}} =
    -s_{1}c_{2}-c_{1}s_{2} = -s_{12}
\end{equation}

The linear Jacobian, calculated from the partial derivates, is:

\begin{equation}
  \prescript{0}{}J_{V} = \begin{bmatrix}
    -\sqrt{2}s_{12}c_{3}-c_{12}(s_{3}-1)-s_{1} & -\sqrt{2}s_{12}c_{3}-c_{12}(s_{3}-1) & -\sqrt{2}c_{12}s_{3}-s_{12}c_{3} & 0 \\
    \sqrt{2}c_{12}c_{3}-s_{12}(s_{3}-1)+c_{1}  & \sqrt{2}c_{12}c_{3}-s_{12}(s_{3}-1)  & -\sqrt{2}s_{12}s_{3}+c_{12}s_{3} & 0 \\
    0                                          & 0                                    & c_{3}                            & 0 \\
                         \end{bmatrix}    
\end{equation}

For $q=[0, 90^{\circ}, -90^{\circ}, 0]^{T}$, the Jacobian is the composition of the linear calculated above and the angular:

\begin{equation}
  \prescript{0}{}J = \begin{bmatrix}
    0 & 0 & 0                  & 0 \\
    3 & 2 & \sqrt{2}           & 0 \\
    0 & 0 & 0                  & 0 \\
    0 & 0 & \frac{\sqrt{2}}{2} & \frac{\sqrt{2}}{2} \\
    0 & 0 & 0                  & 0 \\
    1 & 1 & \frac{\sqrt{2}}{2} & \frac{\sqrt{2}}{2} \\
                         \end{bmatrix}
\end{equation}

\part

A general force is a vector describing the force and the torque applied at a given point. It is similar to the general coordinates of a solid body. The general force vector is specific to a robot link and can be translated using the rotation part of the Jacobian. This is not very obvious, just consider that the force intensity is not modified by the frame translation.

\begin{equation}
  \prescript{0}{}F = \begin{bmatrix}
    \prescript{0}{4}R & 0 \\
    0 & \prescript{0}{4}R \\
                     \end{bmatrix}
  \prescript{4}{}F
\end{equation}

The rotation matrix has to be calculated for the given position:

\begin{equation}
  \prescript{0}{4}R = \begin{bmatrix}
    \frac{\sqrt{2}}{2}  & 0 & \frac{\sqrt{2}}{2} \\
    0                   & 1 & 0                  \\
    -\frac{\sqrt{2}}{2} & 0 & \frac{\sqrt{2}}{2} \\
                      \end{bmatrix}
\end{equation}

So in frame $\{0\}$, the general force is described by $\begin{bmatrix}0, 6, 0, 10.607, 0, 0.707\end{bmatrix}^{T}$.

The torque required to handle this force is given by:

\begin{equation}
  \tau = J^{T}F
\end{equation}

\begin{equation}
  J^{T} = \begin{bmatrix}
    0 & 3        & 0 & 0                  & 0 & 1                  \\
    0 & 2        & 0 & 0                  & 0 & 1                  \\
    0 & \sqrt{2} & 0 & \frac{1}{\sqrt{2}} & 0 & \frac{1}{\sqrt{2}} \\
    0 & 0        & 0 & \frac{1}{\sqrt{2}} & 0 & \frac{1}{\sqrt{2}} \\
          \end{bmatrix}
\end{equation}

Applied to the general force determine in frame $\{0\}$, that gives $\begin{bmatrix}18.707, 12.707, 16.485\end{bmatrix}^{T}$. 

\end{parts}

\question

\begin{parts}

\end{parts}

\end{questions}

\end{document}

