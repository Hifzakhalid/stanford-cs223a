%insert copyright here
\documentclass{exam}
\usepackage{mathtools}
\begin{document}
\title{Stanford CS223A - Introduction to robotics \\
  Homework \#4}
\author{Arn-O}
\date{February 2014}
\maketitle
 
\begin{questions}

\question

\begin{parts}

\part

The linear Jacobian is derivated from the vector position. To be noticed the following trigonometric properties:

\begin{equation}
  \frac{\partial s_{12}}{\partial \theta_{1}} =
    c_{1}c_{2}-s_{1}s_{2} = c_{12}
\end{equation}

And:

\begin{equation}
  \frac{\partial c_{12}}{\partial \theta_{1}} =
    -s_{1}c_{2}-c_{1}s_{2} = -s_{12}
\end{equation}

The linear Jacobian, calculated from the partial derivates, is:

\begin{equation}
  \prescript{0}{}J_{V} = \begin{bmatrix}
    -\sqrt{2}s_{12}c_{3}-c_{12}(s_{3}-1)-s_{1} & -\sqrt{2}s_{12}c_{3}-c_{12}(s_{3}-1) & -\sqrt{2}c_{12}s_{3}-s_{12}c_{3} & 0 \\
    \sqrt{2}c_{12}c_{3}-s_{12}(s_{3}-1)+c_{1}  & \sqrt{2}c_{12}c_{3}-s_{12}(s_{3}-1)  & -\sqrt{2}s_{12}s_{3}+c_{12}s_{3} & 0 \\
    0                                          & 0                                    & c_{3}                            & 0 \\
                         \end{bmatrix}    
\end{equation}

For $q=[0, 90^{\circ}, -90^{\circ}, 0]^{T}$, the Jacobian is the composition of the linear calculated above and the angular:

\begin{equation}
  \prescript{0}{}J = \begin{bmatrix}
    0 & 0 & 0                  & 0 \\
    3 & 2 & \sqrt{2}           & 0 \\
    0 & 0 & 0                  & 0 \\
    0 & 0 & \frac{\sqrt{2}}{2} & \frac{\sqrt{2}}{2} \\
    0 & 0 & 0                  & 0 \\
    1 & 1 & \frac{\sqrt{2}}{2} & \frac{\sqrt{2}}{2} \\
                         \end{bmatrix}
\end{equation}

\part

A general force is a vector describing the force and the torque applied at a given point. It is similar to the general coordinates of a solid body. The general force vector is specific to a robot link and can be translated using the rotation part of the Jacobian. This is not very obvious, just consider that the force intensity is not modified by the frame translation.

\begin{equation}
  \prescript{0}{}F = \begin{bmatrix}
    \prescript{0}{4}R & 0 \\
    0 & \prescript{0}{4}R \\
                     \end{bmatrix}
  \prescript{4}{}F
\end{equation}

The rotation matrix has to be calculated for the given position:

\begin{equation}
  \prescript{0}{4}R = \begin{bmatrix}
    \frac{\sqrt{2}}{2}  & 0 & \frac{\sqrt{2}}{2} \\
    0                   & 1 & 0                  \\
    -\frac{\sqrt{2}}{2} & 0 & \frac{\sqrt{2}}{2} \\
                      \end{bmatrix}
\end{equation}

So in frame $\{0\}$, the general force is described by $\begin{bmatrix}0, 6, 0, 10.607, 0, 0.707\end{bmatrix}^{T}$.

The torque required to handle this force is given by:

\begin{equation}
  \tau = J^{T}F
\end{equation}

\begin{equation}
  F = -F_{app}
\end{equation}

\begin{equation}
  J^{T} = \begin{bmatrix}
    0 & 3        & 0 & 0                  & 0 & 1                  \\
    0 & 2        & 0 & 0                  & 0 & 1                  \\
    0 & \sqrt{2} & 0 & \frac{1}{\sqrt{2}} & 0 & \frac{1}{\sqrt{2}} \\
    0 & 0        & 0 & \frac{1}{\sqrt{2}} & 0 & \frac{1}{\sqrt{2}} \\
          \end{bmatrix}
\end{equation}

Applied to the general force determine in frame $\{0\}$, that gives $-\begin{bmatrix}18.707, 12.707, 16.485, 8.0\end{bmatrix}^{T}$. 

\part

The equations used to solve this problem are the ones from the propagation, see p.17 of the handout 4. For a given segment:

\begin{subequations}
  \begin{equation}
    f_{i} = f_{i+1}
  \end{equation}
  \begin{equation}
    n_{i} = n_{i+1} + P_{i+1} \times f_{i+1}
  \end{equation}
\end{subequations}

Mind the convention of the schematic, especially for the sign (and keep in mind that the forces should be balanced). 

So the force part of the generalized force vector is:

\begin{equation}
  F = F_{4} =  -\begin{bmatrix}0, 6, 0\end{bmatrix}^{T}
\end{equation}

For the torque, we need to convert the position vector into a skew matrix to apply the cross product. The skew matrix of a displacement of 9 units along the $\hat Z$ axis is:

\begin{equation}
  P = \begin{bmatrix}
    0 & -9 & 0 \\
    9 & 0  & 0 \\
    0 & 0  & 0 \\
      \end{bmatrix}
\end{equation}

\begin{equation}
  P \times F = \begin{bmatrix}54, 0, 0\end{bmatrix}^T  
\end{equation}

\begin{equation}
  N = N_{4} - P \times F
\end{equation}

Given that $N_{4} = -\begin{bmatrix}7, 0, 8\end{bmatrix}$, the torque part of the generalized force at the end of the screwdriver is $-\begin{bmatrix}61, 0, 8\end{bmatrix}$.

\end{parts}

\question

\begin{parts}

\part

The workspace of this manipulator is a cylinder of diameter $L_{2}+L_{3}$, excluding the inner cylinder of diameter $L_{2}-L_{3}$. No specific dimensions are given for the height. 

\part

The dexterous workspace is the space with all degrees of freedom. This manipulator has no such workspace. 

\part

As the prismatic joints are moving along the same axis, there are infinite solutions. If one is actuated up, the other is actuated down for the same position.

\part

Without the first prismatic joint, there is only one degree of freedom for the height. That means there are two solutions, as any position in the plan can be reached by two revolute joint positions.

\part

In this last configuration, there is only one position for the height and infinite solutions for the position in the plan.

\end{parts}

\question

\begin{parts}

\part

The cubic equation for this joint position is:

\begin{equation}
  \theta = a_{0} + a_{1}t + a_{2}t^{2} + a_{3}t^{3}
\end{equation}

The derivate of this function is:

\begin{equation}
  \dot \theta = a_{1} + 2a_{2}t + 3a_{3}t^{2}
\end{equation}

The initial and final conditions can now be applied. For $t=0$:

\begin{itemize}
  \item $\theta=\theta_{0}$ so $a_{0}=\theta_{0}$
  \item $\dot \theta=0$ so $a_{1}=0$
\end{itemize}

For $t=t_{f}$:

\begin{itemize}
  \item $\theta_{f} = \theta_{0} + a_{2}t_{f}^{2} + a_{3}t_{f}^{3}$
  \item $0 = 2a_{2}t_{f} + 3a_{3}t_{f}^{2}$
\end{itemize}

The last two equations are enough to solve $a_{2}$ and $a_{3}$:

\begin{equation}
  a_{2} = \frac{3(\theta_{f} - \theta_{0})}{t_{f}^2}
\end{equation}

And:

\begin{equation}
  a_{3} = \frac{-2(\theta_{f} - \theta_{0})}{t_{f}^{3}}
\end{equation}

\part

Given that the joint position is a cubic curve, and that the derivate start and end values are zero, it means that there is one maximum in between (let's consider that the joint position value is positive). At this point, the second derivative is zero. Moreover, the first derivative reaches its maximum value.

\begin{equation}
  \dot \theta = \frac{6(\theta_{f} - \theta_{0})}{t_{f}^2}t + \frac{-6(\theta_{f} - \theta_{0})}{t_{f}^3}t^{2}
\end{equation} 

And:

\begin{equation}
  \ddot \theta = \frac{6(\theta_{f} - \theta_{0})}{t_{}f^2} + \frac{-12(\theta_{f} - \theta_{0})}{t_{f}^3}t
\end{equation}

So at the optimal point:

\begin{subequations}
  \begin{equation}
    6t_{f}(\theta_{f} - \theta_{0}) - 12(\theta_{f} - \theta_{0})t = 0
  \end{equation}
  \begin{equation}
    t_{f} - 2t = 0
  \end{equation}
  \begin{equation}
    t = \frac{t_{f}}{2}
  \end{equation}
\end{subequations}

Applied to the joint value first derivative constraint:

\begin{subequations}
  \begin{equation}
    \dot \theta_{max} > \frac{6(\theta_{f} - \theta_{0})}{t_{f}^2} \times \frac{t_{f}}{2} + \frac{-6(\theta_{f} - \theta_{0})}{t_{f}^3} \times \frac{t_{f}^{2}}{4}
  \end{equation}
  \begin{equation}
    4t_{f}\dot\theta_{max} > 6(\theta_{f} - \theta_{0})
  \end{equation}
  \begin{equation}
    t_{f} > \frac{3(\theta_{f} - \theta_{0})}{2\dot\theta_{max}}
  \end{equation}
\end{subequations}

\part

The second derivative of the joint position is a straight line between $\ddot \theta_{max}$ and $-\ddot \theta_{max}$ (based on the configuration described above). The constraint from the second derivative can be determined at $t=0$.

\begin{subequations}
  \begin{equation}
    \ddot \theta_{max} > \frac{6(\theta_{f} - \theta_{0})}{t_{f}^2}
  \end{equation}
  \begin{equation}
    t_{f} > \sqrt{ \frac{6(\theta_{f} - \theta_{0})}{\ddot \theta_{max}} }
  \end{equation}
\end{subequations}

\end{parts}

\end{questions}

\end{document}

