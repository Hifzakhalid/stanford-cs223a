%insert copyright here
\documentclass{exam}
\usepackage{mathtools}
\begin{document}
\title{Stanford CS223A - Introduction to robotics \\
  Homework \#4}
\author{Arn-O}
\date{February 2014}
\maketitle

\begin{questions}

\question

\begin{parts}

\part

The linear Jacobian is derivated from the vector position. To be noticed the following trigonometric properties:

\begin{equation}
  \frac{\partial s_{12}}{\partial \theta_{1}} =
    c_{1}c_{2}-s_{1}s_{2} = c_{12}
\end{equation}

And:

\begin{equation}
  \frac{\partial c_{12}}{\partial \theta_{1}} =
    -s_{1}c_{2}-c_{1}s_{2} = -s_{12}
\end{equation}

The linear Jacobian, calculated from the partial derivates, is:

\begin{equation}
  \prescript{0}{}J_{V} = \begin{bmatrix}
    -\sqrt{2}s_{12}c_{3}-c_{12}(s_{3}-1)-s_{1} & -\sqrt{2}s_{12}c_{3}-c_{12}(s_{3}-1) & -\sqrt{2}c_{12}s_{3}-s_{12}c_{3} & 0 \\
    \sqrt{2}c_{12}c_{3}-s_{12}(s_{3}-1)+c_{1}  & \sqrt{2}c_{12}c_{3}-s_{12}(s_{3}-1)  & -\sqrt{2}s_{12}s_{3}+c_{12}s_{3} & 0 \\
    0                                          & 0                                    & c_{3}                            & 0 \\
                         \end{bmatrix}    
\end{equation}

For $q=[0, 90^{\circ}, -90^{\circ}, 0]^{T}$, the Jacobian is the composition of the linear calculated above and the angular:

\begin{equation}
  \prescript{0}{}J = \begin{bmatrix}
    0 & 0 & 0                  & 0 \\
    3 & 2 & \sqrt{2}           & 0 \\
    0 & 0 & 0                  & 0 \\
    0 & 0 & \frac{\sqrt{2}}{2} & \frac{\sqrt{2}}{2} \\
    0 & 0 & 0                  & 0 \\
    1 & 1 & \frac{\sqrt{2}}{2} & \frac{\sqrt{2}}{2} \\
                         \end{bmatrix}
\end{equation}

\end{parts}

\question

\begin{parts}

\end{parts}

\end{questions}

\end{document}

